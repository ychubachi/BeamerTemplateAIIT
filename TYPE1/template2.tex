% Created 2014-03-19 水 16:23
\documentclass[dvipdfmx]{beamer}
\usepackage{bxdpx-beamer}
\usepackage{pxjahyper}
\usepackage{minijs}
\renewcommand{\kanjifamilydefault}{\gtdefault}
\AtBeginSection[]
{
  \begin{frame}<beamer>{Outline}
  \tableofcontents[currentsection,currentsubsection]
  \end{frame}
}
\usetheme{AIIT0}
\author{産業技術大学院大学 \linebreak 中鉢欣秀}
\date{2014-02-06}
\title{私がPBL(Project Based Learning)で学んだこと}
\hypersetup{
  pdfkeywords={},
  pdfsubject={},
  pdfcreator={Emacs 24.3.2 (Org mode 8.2.5h)}}
\begin{document}

\maketitle
\begin{frame}{Outline}
\tableofcontents
\end{frame}


\section{はじめに}
\label{sec-1}
\begin{frame}[label=sec-1-1]{自己紹介}
\begin{center}
\begin{tabular}{lll}
1991年 & 4月 & 慶應義塾大学環境情報学部 入学\\
1995年 & \alert{10月} & 同大大学院 政策・メディア研究科 修士課程 入学\\
1997年 & 10月 & 同大大学院 政策・メディア研究科 博士課程 入学\\
 &  & 合資会社ニューメリック設立,*社長就任*\\
2004年 & 10月 & 慶應義塾大学政策メディア研究科後期博士課程卒\\
2005年 & 4月 & 独立行政法人科学技術振興機構 PD級研究員\\
 &  & (長岡技術科学大学)\\
2006年 & 4月 & 産業技術大学院大学 産業技術研究科\\
 &  & 情報アーキテクチャ専攻 准教授\\
\end{tabular}
\end{center}
\end{frame}
\begin{frame}[label=sec-1-2]{河合先生からの依頼事項}
\begin{block}{大学院特別講義}
PBLによるソフトウェア工学教育
\end{block}
\begin{block}{話してほしい内容}
CSの基本は、全員、学部時代にたたきこまれているのですが、まっとうな
(=工業的な)ソフトウェア開発を行なったことのある学生は少数派です。

ですので、例えば、複数人のプロジェクトとして、ソフトウェアをコスト等各
種制限のあるなかで作成することのむずかしさと、その実際の対策、そして、
そうしたことを貴大学院のPBLのなかでは、どのように教えていらっしゃる
のかを、できるだけ具体的にご講演いただけるとうれしく思います。
\end{block}
\end{frame}
\section{合資会社ニューメリック}
\label{sec-2}
\begin{frame}[label=sec-2-1]{ニューメリック誕生の背景}
\begin{block}{慶應義塾大学大岩研究室}
ソフトウエア工学や、技術者教育に関心のある学生が集った研究室。
愛称は「Creative Workspace: CreW」。
\end{block}
\begin{block}{1997年の状況}
\begin{itemize}
\item Windows 95が普及
\item まだ、初期状態でインターネットは使えなかった
\item ライブラリのインストールが必要
\end{itemize}
\end{block}
\end{frame}
\begin{frame}[label=sec-2-2]{アルバイトから会社設立へ}
\begin{block}{ベンチャーブーム}
\begin{itemize}
\item 少しプログラムがかければ仕事はいくらでもあった
\item 後輩が参加したベンチャーの仕事を手伝って面白かった
\end{itemize}
\end{block}
\begin{block}{会社設立の理由}
\begin{itemize}
\item 「やってみたかった」から
\item 後輩にそそのかされた・笑
\end{itemize}
\end{block}
\begin{block}{会社をやって学んだこと}
\begin{itemize}
\item 実プロジェクトの経験
\item 使える技術
\item お金は簡単には儲からない
\end{itemize}
\end{block}
\end{frame}
\section{実プロジェクトの実例}
\label{sec-3}
\begin{frame}[label=sec-3-1]{AIIT SHOWCASEの構築}
\begin{block}{プロジェクト成果物}
\begin{itemize}
\item \href{http://showcase.aiit.ac.jp/}{AIIT SHOWCASE}
\item 2013-01-29 公開
\end{itemize}
\end{block}

\begin{block}{開発のステップ}
\begin{itemize}
\item 妄想・企画・調査・分析の段階
\item 説得・合意形成・理解・仲間集めの段階
\item 具体化・実装・成長の段階
\item 運用・手入れの段階
\end{itemize}
\end{block}
\end{frame}

\section{PBLとは?}
\label{sec-4}
\begin{frame}[label=sec-4-1]{PBLってなんだろう?}
\begin{block}{PBLの流行}
\begin{itemize}
\item 慶應SFC「コラマネ」
\begin{itemize}
\item 現代GP
\end{itemize}
\item あちらこちらでPBL
\begin{itemize}
\item AIITでもPBL
\end{itemize}
\end{itemize}
\end{block}
\begin{block}{私が考えるPBL}
\begin{itemize}
\item PBLは創造の場
\item PBLは教育の環境
\item PBLは挑戦の機会
\end{itemize}
\end{block}
\end{frame}
\begin{frame}[label=sec-4-2]{PBLの限界と次のステップ}
\begin{block}{PBLの限界}
\begin{itemize}
\item Money!
\item Resource!
\item Risk!
\end{itemize}
\end{block}
\begin{block}{PBLのその先に}
\begin{itemize}
\item PBLは3回繰り返すべし!
\begin{itemize}
\item メンバー
\item プロマネ(Scrum Master)
\item クライアント(Product Owner)
\end{itemize}
\item 3回やったら外に飛び出すべし!
\begin{itemize}
\item PBLはやはり箱庭
\end{itemize}
\end{itemize}
\end{block}
\end{frame}
% Emacs 24.3.2 (Org mode 8.2.5h)
\end{document}
